\documentclass{article}
\usepackage[utf8]{inputenc}
\usepackage[margin=1in]{geometry}
\usepackage{amsmath, amssymb, multicol, graphicx}
\usepackage{enumerate, enumitem, dsfont, relsize, wrapfig, bm}
\newcommand\pd{\partial}
\newcommand\bd{\mathbf}
\newcommand\curl{\nabla\times}
\usepackage{dsfont, relsize, wrapfig, bm, float, etoolbox}
\usepackage{wasysym}
\usepackage{array}
\usepackage{booktabs}
\usepackage{listings} %for formatting matlab code
\usepackage{color} %red, green, blue, yellow, cyan, magenta, black, white
\usepackage{subfigure}
\usepackage{tablefootnote}
\usepackage{multirow}
\usepackage{color} %red, green, blue, yellow, cyan, magenta, black, white
\usepackage{listings} %for formatting matlab code
\usepackage[T1]{fontenc}
\usepackage{siunitx}
\usepackage{circuitikz}
\usepackage{bigfoot} % to allow verbatim in footnote

\usepackage{mathtools}
\DeclarePairedDelimiter\ceil{\lceil}{\rceil}
\DeclarePairedDelimiter\floor{\lfloor}{\rfloor}

\definecolor{vgreen}{RGB}{104,180,104}
\definecolor{vblue}{RGB}{49,49,255}
\definecolor{vorange}{RGB}{255,143,102}

\lstdefinestyle{verilog-style}
{
    language=Verilog,
    basicstyle=\small\ttfamily,
    keywordstyle=\color{vblue},
    identifierstyle=\color{black},
    breaklines=true,
    frame=single,
    rulecolor=\color{black},
    commentstyle=\color{vgreen},
    numbers=left,
    numberstyle=\tiny\color{black},
    numbersep=10pt,
    tabsize=4,
    moredelim=*[s][\colorIndex]{[}{]},
    literate=*{:}{:}1,
    postbreak=\mbox{\textcolor{red}{$\hookrightarrow$}\space}
}

\makeatletter
\newcommand*\@lbracket{[}
\newcommand*\@rbracket{]}
\newcommand*\@colon{:}
\newcommand*\colorIndex{%
    \edef\@temp{\the\lst@token}%
    \ifx\@temp\@lbracket \color{black}%
    \else\ifx\@temp\@rbracket \color{black}%
    \else\ifx\@temp\@colon \color{black}%
    \else \color{vorange}%
    \fi\fi\fi
}
\makeatother

\newcommand\cj{\overline}
\newcommand\xhat{\bd{\hat{x}}}
\newcommand\yhat{\bd{\hat{y}}}
\newcommand\rp{\right)}
\newcommand\lp{\left(}
\newcommand\erm{\mathrm{e}}
\newcommand\kohms{\text{ k}\Omega}
\newcommand\muA{\text{ }\mu\text{A}}

\ctikzset{bipoles/length=1cm}
\ctikzset{resistors/scale=0.7, % smaller R
 sources/scale=1,
 capacitors/scale=0.7, % even smaller C
 diodes/scale=0.6, % small diodes
 transistors/scale=1.3} % bigger BJTs


\begin{document}

\begin{titlepage}
   \begin{center}

       \large ECE 111: Advanced Digital Design Project \\
       Prof. Yatish Turakhia
       \vfill

       \LARGE\textbf{Final Project Report} \\
   \vspace{0.8cm}
       \large June 10, 2023 \\

       \vfill

       Conner Hsu (A16665092) \\
       Haozhang Chu (A16484292) \\
       Kirtan Shah (A16227067)

       \date{\today}

   \end{center}
\end{titlepage}

\tableofcontents

\newpage
\section{Simplified SHA-256}

\subsection{Introduction}

% TODO: explain what SHA-256 is

\subsection{Algorithm}

% TODO: explain what SHA-256 algorithm that was implemented.

\subsection{Simulation Results}

% TODO: Waveform, transcript

\newpage
\section{Bitcoin Hashing}

\subsection{Introduction}

% TODO: explain what bitcoin hashing is

\subsection{Algorithm}

% TODO: explain what bitcoin hashing algorithm that was implemented.

\subsection{Resource Usage}

% TODO: resource usage, fitter report snapshot, fmax, timing

\subsection{Simulation Results}

% TODO: Waveform, transcript




\end{document}
